\documentclass[a4paper,11pt]{amsart}
\usepackage{parskip}
\usepackage[T1]{fontenc}
\usepackage[utf8]{inputenc}
\usepackage{graphicx}
\usepackage[left=4.15cm,top=2.10cm,right=4.15cm,bottom=3cm]{geometry}

\begin{document}
\thispagestyle{empty}
\includegraphics[width=2.8cm]{UI-logo} \hfill 1 November 2018\\[0.5cm]

{\bf Reference for Hjalti Þór Ísleifsson}
\bigskip

I first met Hjalti in 2016 when he was a student in a course called
Algebra~I, which is about group and ring theory. When lecturing I tend to
ask the audience a lot of questions and from Hjalti answering some of
those questions I early on realized that he was an extraordinary
student. This, my initial impression, was later borne out by Hjalti's
stellar performance at assignments and at the exam. I also remember
being rather surprised when I learned that Hjalti, in addition to
performing so well, was taking the course one year earlier than is
normal.

Since then I've had the privilege of Hjalti's attendance in another
three courses: Algebra~II (module theory), Algebra~III (Galois theory),
and Enumerative Combinatorics.  These are all demanding courses aimed at
mathematics majors, and Hjalti has continued to perform at the highest
level. Indeed, when it comes to marks he has a perfect score so far.  As
a side note, these courses were all taught in English and, as most
Icelandic students, Hjalti has an excellent command of the English
language.

Hjalti was also instrumental in there being a Combinatorics course at
all. From the first time I met him he has been asking me if I wouldn't
give such a course. I guess he had looked up my research interests on
the web and found that I'm a combinatorics specialist. He further
conducted a poll of the mathematics students which showed that there
were enough students committed to taking the course for it to be
viable. I used that poll as supporting material when I later proposed
the combinatorics course for the department; the course is being taught
for the first time the current semester.

Hjalti is soft spoken, hard working, sharp, original, and as the story
above indicates he is driven. I'm convinced that he will go on to become
a very accomplished mathematician in his own right.

I sum, Hjalti is one of Iceland's finest students, a fact that is also
witnessed by him receiving the Sigurður Helgason award. His application
has my unreserved support.

\ \\[-0.5ex]
Yours sincerely,\medskip\\
\includegraphics[height=1.1cm]{signature}\smallskip\\
Anders Claesson \\
Professor of Mathematics \\
University of Iceland
\end{document}

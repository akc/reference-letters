\documentclass[a4paper,11pt]{amsart}
\usepackage{parskip}
\usepackage[T1]{fontenc}
\usepackage[utf8]{inputenc}

\usepackage{graphicx}
\usepackage[left=4.3cm,top=2.10cm,right=4.3cm,bottom=3cm]{geometry}

\begin{document}
\thispagestyle{empty}
\includegraphics[width=2.6cm]{UI-logo} \hfill 30 November 2020\\[0.5cm]

{\bf Recommendation for Giulio Cerbai}
\bigskip
\bigskip

I know Giulio quite well. We met at a conference in 2018 and we have
been working together on various projects since then. Giulio is
currently a PhD student at the University of Florence and he is expected
to graduate in the spring of 2021. I visited Florence to work with
Giulio and Luca Ferrari (Giulio's advisor) in late 2018. Giulio,
conversely, visited me at UI for a month in 2019. I am currently working
with Giulio on a couple of research projects. He is the driving force of
both projects.

In 1968 Knuth characterized permutations sortable by a stack as those
not containing a certain type of sub-permutation (now called permutation
pattern). This is often seen as the birth of the thriving research field
of permutation patterns, which is the natural home for most of Giulio's
research. One line of research that Giulio has pioneered is an intriguing
twist on Knuth's original problem where the elements on the stack are
required to avoid a permutation pattern. His first paper on this subject
has already received 7 citations since it appeared as a preprint in
July 2019.  Another, even more promising, line of research that Giulio
is pursuing is that of pattern transport theorems on endofunctions. It
is described in detail in Giulio's research proposal.

Giulio has 10 papers that are either published in high quality
international journals or have been submitted to such journals, and I
know that there are several more papers under preparation. This is
deeply impressive for a researcher who is yet to receive his PhD.

I have also been impressed by the speed with which Giulio has become an
expert programmer, an important skill to have in Combinatorics since the
computer is a wonderful ``laboratory'' for experiments with discrete
mathematical structures.

% Giulio's writing skills could be further improved.

My experience of interacting with Giulio could not be more positive.
% He is brilliant, yet modest. He is an independent worker, determined and
% diligent.
I believe he has a bright future as a mathematician and
that he will no doubt make a lasting mark on Combinatorics. Giulio would
be a fantastic addition to the University of Iceland. His application
has my unreserved and enthusiastic support.

Yours sincerely,\medskip\\
\includegraphics[height=1.1cm]{signature}\smallskip\\
Anders Claesson \\
Professor of Mathematics \\
University of Iceland
\end{document}

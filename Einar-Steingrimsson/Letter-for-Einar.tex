\documentclass[a4paper]{amsart}

\setlength{\parindent}{0pt}
\setlength{\parskip}{1ex}

\pagestyle{empty}

\begin{document}
\thispagestyle{empty}

\begin{large}
  {\bf In support of professor\ Einar Steingr\'{\i}msson's application\\
    for rektorship at the University of Iceland}
\end{large}\bigskip

We were Einar's PhD students at the department of mathematics that is
part of both Chalmers University of Technology and the University of
Gothenburg (GU).\footnote{Anders Claesson, 2004, {\it Permutation
    patterns, continued fractions, and a group determined by an ordered
    set}. PhD disertation, Chalmers University of Technology
  (ISBN 91-7291-404-1).\\
  \indent \;\,Sergey Kitaev, 2003, {\it Generalized Patterns in Words
    and Permutations}. PhD disertation, University of Gothenburg (ISBN
  91-628-5521-2).}

Since then we have built successful research careers and Einar has
played a key role in many of our achievements. In 2003 to 2005, after
graduating from Chalmers/GU, we have jointly held positions at three
universities and one research institute: Linnaeus University
(previously, Kalmar University), University of California at San Diego,
University of Kentucky, and the Mittag-Leffler Institute (Royal Swedish
Academy of Sciences).

In 2005 we both chose to take up positions at Reykjavik University
(RU), where Einar had just started to build BSc and MEd programs in
mathematics (from scratch) and what became a world-leading
combinatorics group.

Most of the group's members migrated to the United Kingdom in 2011, and
the group is now known as the {\em Strathclyde Combinatorics Group},
whose leader is Einar.

We shall now discuss Einar's roles as our advisor at Chalmers/GU and as
our supervisor at RU.

{\bf Einar as our advisor at Chalmers/GU.}  Einar was always very
supportive, which was especially important to the second author for whom
PhD studies in Sweden also was his first trip abroad.  Einar followed our
work closely, encouraging us to tackle substantial problems from the
very beginning. He gave us much freedom in choosing problems which we
found interesting to tackle but made sure that we didn't waste our time
on unrealistic pursuits.

Einar was (and still is!) a master at setting high expectations while at
the same time creating a fun and relaxed atmosphere of equals. This is a
priceless gift for creative work.  Einar gave us frequent feedback on
our work and constantly pointed out new directions. He was very
supportive in every aspect of our work and was always eager to spend
considerable time helping us along.

% This was especially important once it came to proofreading our texts.
%  ---
% being not native speakers, we often needed Einar's help with English;
% also, we needed help with learning style to write mathematical papers.

Einar guided us into the international research community in a highly
successful way.  He was efficient in recommending conferences for us to
attend and encouraging us to get to know people from whom we could learn
new things relevant to our work. He also helped arrange so that we could
visit world leading researchers abroad for longer periods of time.

Einar managed to build one of the strongest and most productive research
groups in the large mathematics department at Chalmers/GU, even though
he was the only faculty member in the group. Einar also created one of
the most popular seminar series in the department, where we were giving
talks regularly, and where we hosted combinatorial talks by a steady stream of visitors from outside of Chalmers/GU.

{\bf Einar as our supervisor at RU.} Einar had clear and ambitious
expectations for our teaching. He would always make sure, by protecting
us from too much other work, that we could live up to those expectations
and simultaneously pursue our research in a successful way.

Einar expressed clear ideas about what our teaching should accomplish
for the students, but let us organize our teaching in a way that suited
our styles. He discussed this often and followed closely how things were
going, encouraging novel approaches and experimenting.

Einar led the construction of educational programs that made strong
demands on the students, who rated the math teachers in these programs
far above the average of the university year after year.  While
dedicating much of his energy to administration and teaching, Einar,
again, managed to build a very strong research group, attracting many
outstanding postdocs from all over the world, thus creating an
outstanding environment for our research and making
the group highly visible internationally.

In brief, Einar is a confident and exceptionally competent person in
whichever role he is playing at a university.  He has a clear vision and
knows on a very deep level how to create success in teaching and
research.  Finally, Einar is an exceedingly good administrator.  His
candidature for the rektorship has our unreserved support.

\ \\
\ \\

\noindent
Dr Anders Claesson and Dr Sergey Kitaev \hfill 23-02-2015\\
Department of Computer and Information Sciences\\
University of Strathclyde
\end{document}

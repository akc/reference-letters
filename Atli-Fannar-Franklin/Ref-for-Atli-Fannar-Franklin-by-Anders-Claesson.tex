\documentclass[a4paper,11pt]{amsart}
\usepackage{parskip}
\usepackage[T1]{fontenc}
\usepackage[utf8]{inputenc}
\usepackage{graphicx}
\usepackage[left=4.15cm,top=2.10cm,right=4.15cm,bottom=3cm]{geometry}

\begin{document}
\thispagestyle{empty}
\includegraphics[width=2.8cm]{UI-logo} \hfill 23 January 2020\\[0.5cm]

{\bf Reference for Atli Fannar Franklín}
\bigskip

I have had the privilege of Atli's attendance in two of my courses, namely
Algebraic Structures (2017) and Enumerative Combinatorics (2018).  These
are demanding courses aimed at mathematics majors, and yet Atli received
the highest possible mark in both courses. The Enumerative Combinatorics
course is particularly challenging. It is largely about combinatorial
species, a category theory approach to combinatorics, which is usually a
postgraduate level topic. As a side note, both courses were taught in
English and, as most Icelandic students, Atli has a good command of the
English language.

I was also Atli's advisor on his BSc project. It is known that the
generating function for the Fubini numbers, $F_n$, admits a simple
continued fraction expansion:\\[-3ex]
\begin{center}
  \includegraphics[height=3cm]{Fubini}
\end{center}
In his BSc project Atli gave the first combinatorial proof if this
result. We, of course, do not demand that our students produce new
results or new proofs of old results in their BSc projects, but Atli did
so nonetheless. Now, I should mention that Atli has a weakness and that
is his writing. He struggles a bit with presenting his ideas clearly in
text and this is evident when reading his BSc thesis. I am confident
that he will be able to improve a lot by simply practicing more.

In sum, Atli is a very fine student, a fact that is also witnessed by
him receiving the Sigurður Helgason award. His application has my
enthusiastic support.

\ \\[-0.5ex]
Yours sincerely,\medskip\\
\includegraphics[height=1.1cm]{signature}\smallskip\\
Anders Claesson \\
Professor of Mathematics \\
University of Iceland
\end{document}

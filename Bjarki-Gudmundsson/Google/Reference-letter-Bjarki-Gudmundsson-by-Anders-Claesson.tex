\documentclass[a4paper,11pt]{amsart}
\usepackage{parskip}
\usepackage[T1]{fontenc}
\usepackage[utf8]{inputenc}
\usepackage{graphicx}
\usepackage[left=4.15cm,top=2.10cm,right=4.15cm,bottom=3cm]{geometry}

\begin{document}
\thispagestyle{empty}
\includegraphics[width=2.8cm]{UI-logo} \hfill 14 November 2019\\[0.3cm]

{\bf Reference for Bjarki Ágúst Guðmundsson}
\bigskip

I know Bjarki very well. We first met in 2013 and I've been working with
him on various projects since then. I was his MSc supervisor on a
project about formalizing---in such detail as to be machine
verifiable---the so called translation method in enumerative
combinatorics, which is a way to semi-automatically translate an
algebraic proof of equinumeracy to a bijective proof of the same
fact. Bjarki implemented the translation method in Agda, a dependently
typed functional programming language in which programs are proofs and
vice versa. This was an impressive achievement.

More recently we have been working together on a project concerning
permutations sortable by a so called pop-stack. This resulted in a paper
that has been published in Advances in Applied Mathematics; it was also
accepted to FPSAC, the premier conference on enumerative and
algebraic combinatorics. Moreover, we have produced a preprint (joint
with Jay Pantone) on a related topic that I expect will be very well
received.

Bjarki is an exceptional student and programmer. During his
undergraduate studies at Reykjavik University he was on the dean's list
every semester he was eligible, and graduated with the highest GPA. He
was then awarded the Alan Turing Scholarship, a full scholarship given
to strong students applying for master's studies at Reykjavik
University. Bjarki is the coauthor of several research papers.  His
papers have been presented at the Joint Mathematics Meetings (2015), the
16th Italian Conference on Theoretical Computer Science, the Permutation
Patterns Conference (2015 and 2016), and at the MIT Combinatorics
Seminar (2013). If Bjarki wanted to, he could definitely get a PhD in
mathematics or in computer science.

Bjarki is a very successful competitive programmer (Iceland's best). He
has competed in numerous contests in different parts of the world and he
has organized some contests as well. Bjarki has also held a course
(T-414-ÁFLV) about competitive programming at Reykjavik University.

In summary, my experience of interacting with Bjarki couldn't be more
positive; he's curious, intelligent and a very mature and independent
worker for such an early stage in his career. He easily handles big work
loads; in fact, he seems to thrive under pressure. His application to
Google has my unreserved and enthusiastic support.

\ \\[-0.5ex]
Yours sincerely,\medskip\\
\includegraphics[height=1.1cm]{signature}\smallskip\\
Anders Claesson \\
Professor of Mathematics \\
University of Iceland
\end{document}

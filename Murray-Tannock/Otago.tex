\documentclass[a4paper,10pt]{amsart}
\usepackage{parskip}

\begin{document}
\thispagestyle{empty}
\ \\
{\bf Recommendation for Murray Tannock}\hfill 4 August 2016
\bigskip
\bigskip
\bigskip

I've known Murray since the spring of 2015 when he started working on
his M.Sc.\ under the supervision of Henning Ulfarsson at Reykjavik
University. I was one of two opponents when Murray successfully
defended his M.Sc.\ thesis in late May this year. I'm currently involved
in a project with Murray, Henning and Christian Bean regarding a new
approach to enumerating (some sub-classes of) 4231-avoiding
permutations.

My experience of interacting with Murray could hardly be more positive;
he has a likable personality and a good sense of humor, he is an
intelligent and independent worker.

Murray has already produced one research paper (with coathors) on
``Pattern avoiding permutations and independent sets in graphs''.  I'm
confident that another paper can be distilled from his thesis work. This
is quite impressive at this early stage in his career. In addition,
Murray has proven to be an expert programmer (in Python/Sage and Julia)
and this has helped him/us a great deal, e.g., when testing conjectures.

Murray has given talks at three conferences: Permutation Patterns 2015
and 2016, and the British Combinatorial Conference 2015. I was in the
audience for two of these talks and both were well delivered and well
received.

Murray has an undergraduate degree (B.Sc.) from University of St~Andrews
(2010--2014). I lectured at the University of Strathclyde (Glasgow) from
2011 to 2015. During those 5 years I taught hundreds of undergraduate
students, but I didn't meet anyone of Murray's calibre; neither with
regard to mathematical abilities, nor (more surprisingly, perhaps) with
regard to programming proficiency. When making this comparison one
should keep in mind that I was at a computer science department, rather
than a mathematics department, and that Strathclyde is ranked a bit
lower than St Andrews (and thus may not attract the very best students),
but I still think that this reflects very well on Murray's abilities and
potential.

Clearly Murray is a talented young mathematician and computer scientist
and undoubtedly he would benefit greatly from doing a PhD under
the supervision of Michael Albert. I strongly support his application.



\ \\
Yours sincerely,\smallskip\\
Professor Anders Claesson \\
University of Iceland
\end{document}

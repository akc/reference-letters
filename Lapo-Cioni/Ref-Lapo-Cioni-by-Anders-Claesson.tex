\documentclass[a4paper,11pt]{amsart}
\usepackage{parskip}
\usepackage[T1]{fontenc}
\usepackage[utf8]{inputenc}

\usepackage{graphicx}
\usepackage[left=4.3cm,top=2.10cm,right=4.3cm,bottom=3cm]{geometry}

\newcommand{\Av}{\mathrm{Av}}
\newcommand{\B}{\mathbf{B}}
\newcommand{\Q}{\mathbf{Q}}
\renewcommand{\S}{\mathbf{S}}
\newcommand{\unB}{{\B^{-1}}}
\newcommand{\unQ}{{\Q^{-1}}}
\newcommand{\unS}{{\S^{-1}}}

\begin{document}
\thispagestyle{empty}
\includegraphics[width=2.6cm]{UI-logo} \hfill 30 November 2022\\[0.5cm]

{\bf In support of Lapo Cioni's application}
\bigskip
\bigskip

I don't know Lapo personally, but I do know his work; I am an examiner
of Lapo's PhD thesis. His research subject sits in the intersection of
computer science and mathematics. It's about sorting using primitive
devices, or data-structures, and it's about the structure of classes of
permutations.

One reason for being interested in preimages of classes is to
understand compositions of sorting operators. If $T$ and $U$ are
sorting operators, then the permutations sortable by $T\circ U$ are
$(T\circ U)^{-1}(\Av(21)) = U^{-1}(T^{-1}(\Av(21)))$. For instance, let $S$ be
the stack sorting operator.

Knuth observed that $\unS(\Av(21)) = \Av(231)$. Consequently, the set
$(\S^2)^{-1}(\Av(21))$ of permutations sortable by two passes through a
stack is $\unS(\Av(231))$. Similarly, the permutations sortable by
$\S\circ \B$ are $\unB(\Av(231))$, and the permutations sortable by
$\B\circ \S$ are $\unS(\Av(231, 321))$.


Lapo is obviously a very talented young mathematician. I strongly support
his application.

Yours sincerely,\medskip\\
\includegraphics[height=1.1cm]{signature}\smallskip\\
Anders Claesson \\
Professor of Mathematics \\
University of Iceland
\end{document}

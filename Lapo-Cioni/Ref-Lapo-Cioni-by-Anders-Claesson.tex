\documentclass[a4paper,11pt]{amsart}
\usepackage{parskip}
\usepackage[T1]{fontenc}
\usepackage[utf8]{inputenc}

\usepackage{graphicx}
\usepackage[left=3.4cm,top=2.1cm,right=3.4cm,bottom=2cm]{geometry}

\newcommand{\Av}{\mathrm{Av}}
\newcommand{\B}{\mathbf{B}}
\newcommand{\Q}{\mathbf{Q}}
\renewcommand{\S}{\mathbf{S}}
\newcommand{\unB}{{\B^{-1}}}
\newcommand{\unQ}{{\Q^{-1}}}
\newcommand{\unS}{{\S^{-1}}}
\newcommand{\Id}{\mathrm{Id}}
\newcommand{\Class}{\mathcal{C}}

\begin{document}
\thispagestyle{empty}
\includegraphics[width=2.6cm]{UI-logo} \hfill 30 November 2022\\[0.5cm]

{\bf In support of Lapo Cioni's application}
\bigskip
\bigskip

I don't know Lapo personally, but I do know his work; I am an examiner
of Lapo's PhD thesis. His research subject sits in the intersection of
computer science and mathematics. It's about sorting using primitive
devices, or container data structures, and it's about the structure of
classes of permutations. The starting point of this area can be seen as
Knuth's characterization of permutations sortable using a stack as those
avoiding the pattern 231. Symbolically we may express this as
$\unS(\Id_n)=\Av_n(231)$. A central question in this area and in Lapo's
work is the following: If $T$ is a sorting operator, then how can we
characterize and enumerate the set $T^{-1}(\Id_n)$ of permutations
sortable by $T$; or, a bit more generally, if $\sigma$ is a permutation,
how can we characterize and enumerate the set $T^{-1}(\sigma)$? The
latter formulation can also be seen as a stepping stone to
characterizing $T^{-1}(\Class)$ in which $\Class$ is a \emph{pattern
class}---i.e.\ a downwards closed set of permutations in the containment
order. Typically $T$ is one of $\B$ (bubble-sort), $\S$ (stack-sort) or
$\Q$ (queue-sort). The class $\Class$ can always be taken to be of the
form $\Av(P)$, the set of permutations avoiding the patterns in $P$.

Studying preimages is one of the main avenues for understanding sorting
operators. In particular, they elucidate compositions of sorting
operators: If $T$ and $U$ are sorting operators, then the permutations
sortable by $T\circ U$ are
$(T\circ U)^{-1}(\Av(21)) = U^{-1}(T^{-1}(\Av(21)))$. For instance,
Knuth observed that $\unS(\Av(21)) = \Av(231)$. Consequently, the set
$(\S^2)^{-1}(\Av(21))$ of permutations sortable by two passes through a
stack is $\unS(\Av(231))$. Similarly, the permutations sortable by
$\S\circ \B$ are those in $\unB(\Av(231))$, and the permutations sortable by
$\B\circ \S$ are $\unS(\Av(231, 321))$.

Lapo has, in collaboration with coauthors, been able to completely
describe the sets $\unB(\sigma)$. Indeed, they were able to describe the
whole sorting tree for $\B$. Similarly, Lapo and his advisor have given
a recursive description of the preimages $\unQ(\sigma)$; in particular,
they proved the curious property that $\unQ(\sigma)$ can be of any size
except 3. I another paper Lapo introduces a variant of a queue that he
calls a popqueue. He shows that there is an optimal sorting algorithm
and, again, characterizes preimages.

Departing from the operator perspective for a moment, we also note that
Lapo and coauthors have given an optimal sorting algorithm for the so
called $D^2I$-stack and shown that the corresponding set of sortable
permutations has an infinite basis (the $DI$-stack has a finite basis).

In summary, Lapo has made significant progress on hard problems that are
at the heart of the vibrant research area we know as permutation
pattern. He is obviously a very talented young mathematician, and I
support his application.

Yours sincerely,\medskip\\
\includegraphics[height=1.1cm]{signature}\smallskip\\
Anders Claesson \\
Professor of Mathematics \\
University of Iceland
\end{document}

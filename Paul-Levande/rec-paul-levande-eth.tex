\documentclass[11pt]{amsart}

\begin{document}
\thispagestyle{empty}
\ \\
{\bf Recommendation for Paul Levande}\hfill Dec 9, 2013
\bigskip
\bigskip

\noindent
I know Paul from his work on enumerative aspects of interval orders
and related structures. In March 2010 I received a very polite letter
from Paul in which he claimed to have proved a conjecture by Svante
Linusson and myself. This was an interesting development. I eagerly
read the letter and found the proof to be correct. I wrote back and
congratulated him on his proof. He replied, and told me that he had
also found a proof of a second conjecture in the same paper. This was
the more interesting of the two conjectures. It stated that matchings
on $\{1,\dots,2n\}$ without 2-nestings are enumerated by the so called
Fishburn numbers, $f_n$, whose generating function is
$$
\sum_{n\geq 0}f_nt^n = \sum_{n\geq 1}\prod_{i=1}^n \big(1-(1-t)^i\big).
$$
Zagier, following Stoimenow, had earlier (in work related to
Vassiliev's knot invariants) given a very technical non-combinatorial
proof that the same numbers count non-neighbour-nesting matchings.

The Fishburn numbers also enumerate interval orders. This was proved
by myself and coauthors. Our proof went via a bijection to certain
sequences now known as ascent sequences. The ascent sequences satisfy
a rather obvious recursion. Via that recursion and a very non-obvious
application of the kernel method we managed to do the enumeration.

Paul's proof of the second conjecture started out in the same vein: a
bijection to certain sequences was given. It then took a very
different and much more elegant turn. Instead of relying on
non-combinatorial methods, such as the kernel method, he very cleverly
applied the involution principle. As a testament to the fact that
combinatorial proofs are most often more illuminating than other
proofs, this enabled Paul to refine the enumeration and prove a
conjecture of Kitaev and Remmel. In all, he had managed to prove three
conjectures in one paper, and more importantly he provided a deeper
combinatorial understanding.
\medskip

\noindent
With the caveat that I haven't spent very much time with Paul (I've only
met him at conferences) I'd say that he is a very talented young
mathematician. I strongly support his application.

\ \\
Yours sincerely,\smallskip\\
Anders Claesson \\
Senior Lecturer \\
Department of Computer and Information Sciences \\
University of Strathclyde, UK \\

\end{document}
